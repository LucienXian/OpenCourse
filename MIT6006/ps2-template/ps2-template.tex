%
% 6.006 problem set 2 solutions template
%
\documentclass[12pt,twoside]{article}

\input{macros-sp20}
\newcommand{\theproblemsetnum}{2}

\title{6.006 Problem Set 2}

\begin{document}

\handout{Problem Set \theproblemsetnum}

\setlength{\parindent}{0pt}
\medskip\hrulefill\medskip

{\bf Name:} Your Name

\medskip

{\bf Collaborators:} Name1, Name2

\medskip\hrulefill

%%%%%%%%%%%%%%%%%%%%%%%%%%%%%%%%%%%%%%%%%%%%%%%%%%%%%
% See below for common and useful latex constructs. %
%%%%%%%%%%%%%%%%%%%%%%%%%%%%%%%%%%%%%%%%%%%%%%%%%%%%%

% Some useful commands:
%$f(x) = \Theta(x)$
%$T(x, y) \leq \log(x) + 2^y + \binom{2n}{n}$
% {\tt code\_function}


% You can create unnumbered lists as follows:
%\begin{itemize}
%    \item First item in a list
%        \begin{itemize}
%            \item First item in a list
%                \begin{itemize}
%                    \item First item in a list
%                    \item Second item in a list
%                \end{itemize}
%            \item Second item in a list
%        \end{itemize}
%    \item Second item in a list
%\end{itemize}

% You can create numbered lists as follows:
%\begin{enumerate}
%    \item First item in a list
%    \item Second item in a list
%    \item Third item in a list
%\end{enumerate}

% You can write aligned equations as follows:
%\begin{align}
%    \begin{split}
%        (x+y)^3 &= (x+y)^2(x+y) \\
%                &= (x^2+2xy+y^2)(x+y) \\
%                &= (x^3+2x^2y+xy^2) + (x^2y+2xy^2+y^3) \\
%                &= x^3+3x^2y+3xy^2+y^3
%    \end{split}
%\end{align}

% You can create grids/matrices as follows:
%\begin{align}
%    A =
%    \begin{bmatrix}
%        A_{11} & A_{21} \\
%        A_{21} & A_{22}
%    \end{bmatrix}
%\end{align}

% You can include images and PDFs as follows:
% \includegraphics[width=0.5\textwidth]{img.jpg}

\begin{problems}

\problem  % Problem 1

\begin{problemparts}
\problempart % Problem 1a
T(n) = O(n^2)

\problempart % Problem 1b
T(n) = O(n^4)

\problempart % Problem 1c
T(n) = O(n\log^2n)

\problempart % Problem 1d
T(n) = O(n^2)
\end{problemparts}

\newpage
\problem  % Problem 2

\begin{problemparts}
\problempart % Problem 2a
Because merge sort is not an in-place sort, it is excluded.Next we need a sorting method that minimizes the number of set\_at(). So we use selection sort.

\problempart % Problem 2b
Since comparison is a heavy operation at this time, both selection sorting and insertion sorting need to compare $n^2$ times, and merge sorting only needs to compare $nlogn$ times, so choose merge sorting
\problempart % Problem 2c
The sorting of selection sort and merge sort does not depend on the input, and the insertion sort will exit early when a specific input is entered, so choose insertion sort
\end{problemparts}

\newpage
\problem  % Problem 3
Search inward from either end of the island exponentially and do a binary search

\newpage
\problem  % Problem 4
Use a doubly-linked list to store all undeleted messages and use a sorted array which store a pair of node(viewer, pointer) that the pointer is pointed to the list above.

\newpage
\problem  % Problem 5

\begin{problemparts}
\problempart % Problem 5a
\problempart % Problem 5b
\problempart Submit your implementation to {\small\url{alg.mit.edu}}.
\end{problemparts}

\end{problems}

\end{document}
