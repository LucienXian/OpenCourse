%
% 6.006 problem set 1 solutions template
%
\documentclass[12pt,twoside]{article}

\input{macros-sp20}
\newcommand{\theproblemsetnum}{1}

\title{6.006 Problem Set 1}

\begin{document}

\handout{Problem Set \theproblemsetnum}

\setlength{\parindent}{0pt}
\medskip\hrulefill\medskip

{\bf Name:} LX

\medskip

{\bf Collaborators:} Name1, Name2

\medskip\hrulefill

%%%%%%%%%%%%%%%%%%%%%%%%%%%%%%%%%%%%%%%%%%%%%%%%%%%%%
% See below for common and useful latex constructs. %
%%%%%%%%%%%%%%%%%%%%%%%%%%%%%%%%%%%%%%%%%%%%%%%%%%%%%

% Some useful commands:
%$f(x) = \Theta(x)$
%$T(x, y) \leq \log(x) + 2^y + \binom{2n}{n}$
% {\tt code\_function}


% You can create unnumbered lists as follows:
%\begin{itemize}
%    \item First item in a list
%        \begin{itemize}
%            \item First item in a list
%                \begin{itemize}
%                    \item First item in a list
%                    \item Second item in a list
%                \end{itemize}
%            \item Second item in a list
%        \end{itemize}
%    \item Second item in a list
%\end{itemize}

% You can create numbered lists as follows:
%\begin{enumerate}
%    \item First item in a list
%    \item Second item in a list
%    \item Third item in a list
%\end{enumerate}

% You can write aligned equations as follows:
%\begin{align}
%    \begin{split}
%        (x+y)^3 &= (x+y)^2(x+y) \\
%                &= (x^2+2xy+y^2)(x+y) \\
%                &= (x^3+2x^2y+xy^2) + (x^2y+2xy^2+y^3) \\
%                &= x^3+3x^2y+3xy^2+y^3
%    \end{split}
%\end{align}

% You can create grids/matrices as follows:
%\begin{align}
%    A =
%    \begin{bmatrix}
%        A_{11} & A_{21} \\
%        A_{21} & A_{22}
%    \end{bmatrix}
%\end{align}

% You can include images and PDFs as follows:
% \includegraphics[width=0.5\textwidth]{img.jpg}

\begin{problems}

\problem  % Problem 1

\begin{problemparts}
\problempart % Problem 1a
(f5, f3, f4. f1. f2)
\problempart % Problem 1b
(f1, f2, f5, f4, f3)
\problempart % Problem 1c
 ({f2, f5}, f4, f1, f3)
\problempart % Problem 1d
(f5, f2, f1, f3, f4)
\end{problemparts}

\newpage
\problem  % Problem 2

\begin{problemparts}
\problempart % Problem 2a
python code

\begin{lstlisting}
def reverse(D, i, k):
    if k < 2:
        return
    first = delete_at(i + k - 1)
    second = delete_at(i)
    insert_at(i, second)
    insert_at(i + k - 1, first)
    reverse(D, i + 1, k - 2)
\end{lstlisting}

\problempart % Problem 2b
python code

\begin{lstlisting}
def move(D, i, k, j):
    if k < 1:
        return
    x = delete_at(i)
    if i < j:
        j = j - 1
    insert_at(j, x)
    j = j + 1
    if i > j:
        i = i + 1
    move(D, i, k - 1, j)
\end{lstlisting}
\end{problemparts}

\newpage
\problem  % Problem 3

\newpage
\problem  % Problem 4

\begin{problemparts}
\problempart % Problem 4a
\problempart % Problem 4b
\problempart % Problem 4c
\problempart Submit your implementation to {\small\url{alg.mit.edu}}.
\end{problemparts}

\end{problems}

\end{document}
