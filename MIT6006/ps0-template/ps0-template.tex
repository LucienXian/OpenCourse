%
% 6.006 problem set 0 solutions template
%
\documentclass[12pt,twoside]{article}

\input{macros-sp20}
\newcommand{\theproblemsetnum}{0}

\title{6.006 Problem Set 0}

\begin{document}

\handout{Problem Set \theproblemsetnum}

\setlength{\parindent}{0pt}
\medskip\hrulefill\medskip

{\bf Name:} LX

\medskip\hrulefill

%%%%%%%%%%%%%%%%%%%%%%%%%%%%%%%%%%%%%%%%%%%%%%%%%%%%%
% See below for common and useful latex constructs. %
%%%%%%%%%%%%%%%%%%%%%%%%%%%%%%%%%%%%%%%%%%%%%%%%%%%%%

% Some useful commands:
% $f(x) = \Theta(x)$
% $T(x, y) \leq \log(x) + 2^y + \binom{2n}{n}$
% \ttt{code\_function}


% You can create unnumbered lists as follows:
% \begin{itemize}
%     \item First item in a list
%         \begin{itemize}
%             \item First item in a list
%                 \begin{itemize}
%                     \item First item in a list
%                     \item Second item in a list
%                 \end{itemize}
%             \item Second item in a list
%         \end{itemize}
%     \item Second item in a list
% \end{itemize}

% You can create numbered lists as follows:
% \begin{enumerate}
%     \item First item in a list
%     \item Second item in a list
%     \item Third item in a list
% \end{enumerate}

% You can write aligned equations as follows:
% \begin{align}
%     \begin{split}
%         (x+y)^3 &= (x+y)^2(x+y) \\
%                 &= (x^2+2xy+y^2)(x+y) \\
%                 &= (x^3+2x^2y+xy^2) + (x^2y+2xy^2+y^3) \\
%                 &= x^3+3x^2y+3xy^2+y^3
%     \end{split}
% \end{align}

% You can create grids/matrices as follows:
% \begin{align}
%     A =
%     \begin{bmatrix}
%         A_{11} & A_{21} \\
%         A_{21} & A_{22}
%     \end{bmatrix}
% \end{align}

\begin{problems}

\problem  % Problem 1

\begin{problemparts}
\problempart % Problem 1a
\{6, 12\}
\problempart % Problem 1b
7 % {1, 3, 6, 9, 12, 13, 15}
\problempart % Problem 1c
3 % {1, 9, 13}
\end{problemparts}

\problem  % Problem 2

\begin{problemparts}
\problempart % Problem 2a
1.5
\problempart % Problem 2b
12.25 % E(X)=1/6+2/6+3/6+4/6+5/6+6/6=21/6=7/2, 7/2 * 7/2
\problempart % Problem 2c
\end{problemparts}

\problem  % Problem 3

\begin{problemparts}
\problempart % Problem 3a
true % a≡b(mod m) It is equivalent to dividing a and b by m respectively, and the remainder is the same
\problempart % Problem 3b
false
\problempart % Problem 3c
false
\end{problemparts}

\problem  % Problem 4

Mathematical induction: 

Base case: $n = 1, 1^{3} = 1 = ({\frac{1(1+1)}{2}})^2.$ 

Suppose\ this\ equation\ holds\ for\ n\ =\ k.

When $n = (k+1), \sum_{i=1}^{k+1}i^{3} =  ({\frac{k(k+1)}{2}})^2 (k+1)^{3} = (\frac{k^{2}}{4} + (k+1))(k+1)^{2} = \frac{(k+2)^{2}}{4}(k+1)^2 = (\frac{(k+1)(k+2)}{2})^{2}$.\ This means that when n = k+1, the equation also holds.

\newpage
\problem  % Problem 5

Mathematical induction:
Since the graph G(k+1) can be obtained from the graph G{k} by adding a vertex with degree at most 1, the graph G(k+1) does not contain a circuit too.

\vfill
\problem  % Problem 6
Submit your implementation to {\small\url{alg.mit.edu}}.

\begin{lstlisting}
def count_long_subarray(A):
    '''
    Input:  A     | Python Tuple of positive integers
    Output: count | number of longest increasing subarrays of A
    '''
    count = 0
    ##################
    # YOUR CODE HERE #
    ##################
    return count
\end{lstlisting}

\end{problems}

\end{document}
